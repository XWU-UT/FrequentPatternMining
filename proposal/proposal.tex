%%%%%%%%%%%%%%%%%%%%%%%%%%%%%%%%%%%%%%%%%%%%%%%%%%%%%%%%%%%%%%%%%%%%%%
% COMP 4106
% November 7th, 2012
% Project Proposal
% By Simon Pratt
%%%%%%%%%%%%%%%%%%%%%%%%%%%%%%%%%%%%%%%%%%%%%%%%%%%%%%%%%%%%%%%%%%%%%%
\documentclass[11pt]{article}

\usepackage{geometry}
\geometry{verbose,tmargin=1in,bmargin=1.5in,lmargin=.5in,rmargin=.5in}
\usepackage[pdftex]{graphicx}
\usepackage{fancyhdr}
\usepackage{fix-cm}
\usepackage{amsmath}
\usepackage{enumerate}
\usepackage{amsthm}
\usepackage{amssymb}
\usepackage{parskip}
\usepackage{color}
\usepackage{multicol}
\usepackage{enumitem}
\usepackage{setspace}
\usepackage{newclude} % \include without \clearpage

%%%%%%%%%%%%%%%%%%%%%%%%%%%%%%%%%%%%%%%%%%%%%%%%%%%%%%%%%%%%%%%%%%%%%%
% Extra stuff from John Howat
%%%%%%%%%%%%%%%%%%%%%%%%%%%%%%%%%%%%%%%%%%%%%%%%%%%%%%%%%%%%%%%%%%%%%%

\renewcommand{\implies}{\rightarrow}
\newcommand{\same}{\leftrightarrow}
\newcommand{\AND}{\wedge}
\newcommand{\OR}{\vee}
\newcommand{\cross}{\times}
\newcommand{\xor}{\oplus}
\newcommand{\zz}{\mathbb{Z}}
\newcommand{\solution}[1]{\ifthenelse{\boolean{solutions}}{\textbf{Solution:} #1}{}}
\newcommand{\BigOh}[1]{O\!\left(#1\right)}
\newcommand{\LittleOh}[1]{o\!\left(#1\right)}
\newcommand{\BigOmega}[1]{\Omega\!\left(#1\right)}
\newcommand{\LittleOmega}[1]{\omega\!\left(#1\right)}
\newcommand{\BigTheta}[1]{\Theta\!\left(#1\right)}

%%%%%%%%%%%%%%%%%%%%%%%%%%%%%%%%%%%%%%%%%%%%%%%%%%%%%%%%%%%%%%%%%%%%%%
% Question and Answer environments
% modified from Michiel Smid's code
%%%%%%%%%%%%%%%%%%%%%%%%%%%%%%%%%%%%%%%%%%%%%%%%%%%%%%%%%%%%%%%%%%%%%%

\newcounter{problem}
\newcounter{problempart}

\newcommand{\newProblem}[0]{\stepcounter{problem}{\bf Problem \arabic{problem}}\setcounter{problempart}{0}}
\newcommand{\newPart}[1]{\stepcounter{problempart}{\bf \arabic{problempart}. #1}}

%%%%%%%%%%%%%%%%%%%%%%%%%%%%%%%%%%%%%%%%%%%%%%%%%%%%%%%%%%%%%%%%%%%%%%
% Theorem, Lemma and Definitions
%%%%%%%%%%%%%%%%%%%%%%%%%%%%%%%%%%%%%%%%%%%%%%%%%%%%%%%%%%%%%%%%%%%%%%

\newtheorem*{bthm}{Theorem}
\theoremstyle{definition}
\newtheorem{definition}{Definition}

%%%%%%%%%%%%%%%%%%%%%%%%%%%%%%%%%%%%%%%%%%%%%%%%%%%%%%%%%%%%%%%%%%%%%%
% Fix for amsthm and parskip
% URL: http://tex.stackexchange.com/questions/22119/how-can-i-change-the-spacing-before-theorems-with-amsthm
% From Greg Bint
%%%%%%%%%%%%%%%%%%%%%%%%%%%%%%%%%%%%%%%%%%%%%%%%%%%%%%%%%%%%%%%%%%%%%%

\makeatletter
\def\thm@space@setup{%
  \thm@preskip=\parskip \thm@postskip=0pt
}
\makeatother

%%%%%%%%%%%%%%%%%%%%%%%%%%%%%%%%%%%%%%%%%%%%%%%%%%%%%%%%%%%%%%%%%%%%%%
% Header/footer
% Mostly taken from: 
%  https://texblog.wordpress.com/2007/11/07/headerfooter-in-latex-with-fancyhdr/
%%%%%%%%%%%%%%%%%%%%%%%%%%%%%%%%%%%%%%%%%%%%%%%%%%%%%%%%%%%%%%%%%%%%%%
\pagestyle{fancy}
\fancyhead{}             % clear header
\fancyfoot{}

\fancyhead[L]{COMP4106 - Project Proposal}
\fancyhead[C]{Page \thepage}
\fancyhead[R]{Simon Pratt - 100663987}

%%%%%%%%%%%%%%%%%%%%%%%%%%%%%%%%%%%%%%%%%%%%%%%%%%%%%%%%%%%%%%%%%%%%%%
% Document
%%%%%%%%%%%%%%%%%%%%%%%%%%%%%%%%%%%%%%%%%%%%%%%%%%%%%%%%%%%%%%%%%%%%%%
\begin{document}
\begin{multicols}{2}

\section*{Project Proposal}

\subsection*{Description of the Problem Domain}

The problem domain is to find frequent patterns in large data sets.

\subsection*{Motivation}

Frequent patterns in data can give us insight into the semantic
meaning of the data.  In particular, if we are mining for frequent
patterns in a data set of actions we can identify typical workflow and
anomalous activity, which could be useful for security or process
optimization.

\subsection*{AI Techniques to be Used}

Techniques of frequent pattern mining I would like to consider
include: \texttt{Apriori}, \texttt{FP-growth}, and \texttt{Eclat}
\cite{Han2007,fpmlecture}.

\subsection*{Proposal}

To implement the aforementioned techniques and run them against
\emph{Github}'s data set of actions which has recently been made
public \cite{gharchive}.  This should identify frequent patterns of
events in \emph{Github}'s usage.  The event types in the data are:

\begin{enumerate}
\item CommitCommentEvent
\item CreateEvent
\item DeleteEvent
\item DownloadEvent
\item FollowEvent
\item ForkEvent
\item ForkApplyEvent
\item GistEvent
\item GollumEvent
\item IssueCommentEvent
\item IssuesEvent
\item MemberEvent
\item PublicEvent
\item PullRequestEvent
\item PullRequestReviewCommentEvent
\item PushEvent
\item TeamAddEvent
\item WatchEvent
\end{enumerate}

Patterns of usage among these events should provide some information
about the typical workflow for a social code repository such as
\emph{Github}.

\bibliographystyle{IEEEtran}
\bibliography{refs}

\end{multicols}
\end{document}
